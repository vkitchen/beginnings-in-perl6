\chapter{Quick Tour of Programming}
The hardest part of programming when you're first learning is figuring out what
you can do and how. This section is a quick tour through some simple programs.
It is meant to show you how quickly you can build something that's fun.
Once we've finished here we'll work on refining the concepts and cementing them
in your brain.

The first thing we're going to write is the well known hello world example.
\lstinputlisting{code/tour-hello.p6}
You've now written your first program. Hopefully you've managed to run it,
if not check Appendix A for information on running code.

That first program is pretty boring. Let's make something a bit more interactive
\lstinputlisting{code/tour-prompt.p6}
That's pretty sweet eh? What other questions could you ask?
What about answering the question with a word instead of a number?

Let's break it down to see what's happening. Here we've just stored a number
straight into our variable named \$count. A variable is like a box which we can
store something into. But because we can have lots of boxes storing lots of
things we have to give it a name, just like if we were moving house.
Now we can easily ask for that value back
\lstinputlisting{code/tour-var.p6}
What if we want to store multiple things in our variable?
We use a different type known as an array. Notice it starts with @ instead of
\$ which tells perl we want to be able to use it for multiple values.
We also have to ask for our values back slightly differently.
\lstinputlisting{code/tour-array.p6}
It would be tedious to type out the whole fibonacci series. We invented
computers to solve that exact kind of maths. Here's a program that takes
a list and extends it with the last two items added together, 100 times.
\lstinputlisting{code/tour-fib.p6}
While the fibonacci sequence is undoubtadly cool. It's not what I personally
spend my spare time studying. I much prefer games. Here's a really
easy one you can make. A name generator, in this case for your wolf name

In the output string, we can use curly brackets to tell perl the code needs to
be run. If you're finding perl is just outputting the text it may
be because you're missing the brackets.
\lstinputlisting{code/tour-name-generator.p6}
Another thing we can do in programs is known as conditionals. Simply we can
branch the program to behave in different ways depending on input.
\lstinputlisting{code/tour-if.p6}
One last program to show off some of what we've learnt. Used to love playing
this one as a kid.
\lstinputlisting{code/tour-guessing.p6}
How was that? Not too bad I hope. If you didn't find it too clear you'll start
to get the hang of it with the next sections. All we wanted is for you to be
inspired by what programming can offer.
